\section{Wprowadzenie}
\label{section:wprowadzenie}
Strategia ewolucyjna z adaptacją macierzy kowariancji (ang. Covariance Matrix Adaptation Evolution Strategy, CMA-ES) \cite{cmaes} znajduje się w czołówce metod optymalizacyjnych w ramach konkursów \texttt{IEEE CEC} \cite{cec2017} oraz \texttt{BBOB} \cite{HansenEtal10}. W każdej iteracji algorytm tworzy populację złożoną z punktów wylosowanych przy użyciu wielowymiarowego rozkładu normalnego, którego parametry, tj. wektor wartości oczekiwanej oraz macierz kowariancji, są modyfikowane. Modyfikacja tych parametrów odbywa sie na podstawie reguły CMA (ang. Covariance Matrix Adaptation). Wskutek działaniapowyższej reguły funkcja gęstości prawdopodobieństwa lokalnie aproksymuje funkcję celu i tym samym zwiększa szansę na wylosowanie punktów blisko optimum lokalnego. Dynamika adaptacji rozkładu normalnego sterowana jest również przez kontrolę zasięgu mutacji, która bazuje na mechanizmie ścieżki ewolucyjnej (ang. evolution path) \cite{Hansen2001}. Zwiększona efektywność działania algorytmu wkustek stosowania powyższych mechanizmów jest okupiona znaczącym narzutem obliczeniowym, który często ogranicza zakres praktycznych zastosowań metody.
W artykule tym proponuję modyfikację działania algorytmu \texttt{CMA-ES}, które pozwalają zredukować złożoność obliczeniową metody przy niewielkiej stracie efektywności działania. Proponowane przeze mnie modyfikacje dotyczą reguły adaptacji macierzy kowariancji oraz zasięgu mutacji. Zmiana sposobu sterowania zasięgiem mutacji bazuje na kontroli punktu środkowego populacji \cite{Arabas17}, a zmiany dotyczące adaptacji macierzy kowariancji na pomyśle roważanym przez Beyer'a \cite{SMAES}.
Artykuł ten posiada następującą strukturę. W sekcji (\ref{section:adaptacje}) krótko przedstawiam zasadę działania algorytmu \texttt{CMA-ES} oraz nakreślam problemy związane ze złożonością obliczeniową w jego kanonicznej wersji. Omawiam również dotychczasowe próby rozwiązania tego problemu. Dokładny opis proponowanych przeze mnie metod znajduje się w sekcji (\ref{section:modyfikacje}). Sekcja (\ref{section:eskperymenty}) zawiera komentarz do rezultatów osiągniętych w ramach eksperymentów numerycznych. Sekcja ostatnia, (\ref{section5:podsumowanie}), stanowi podsumowanie artykułu oraz omawiam w niej plan dalszych badań.
